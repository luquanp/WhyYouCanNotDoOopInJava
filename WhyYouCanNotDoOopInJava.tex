\documentclass[11pt,a4paper]{book}
\usepackage[bookmarks,unicode]{hyperref}
\usepackage[top=5cm,bottom=5cm,left=5cm,right=5cm]{geometry}
\usepackage{fontspec,myjapanese,xltxtra,setspace,ascmac,ucs,makeidx,graphicx,color,amsmath,pstricks,multirow,url,wrapfig}
\usepackage{bxokumacro,ulem}
%slashbox,fancyvrb,xunicode,,verbatim,cite,}
\doublespacing
\setmainfont{Adobe Song Std}
\newcommand{\jp}{\fontspec{IPA明朝}}
%\newcommand{\jp}{\fontspec{Droid Sans Japanese}}

\newcommand{\cnh}{\fontspec{Adobe Heiti Std}}
\newcommand{\cns}{\fontspec{Adobe Song Std}}
%\newcommand{\cnhei}{\fontspec{Adobe Heiti Std}}
%\newcommand{\fsa}{\fontspec{FreeSans}}
%\newcommand{\fse}{\fontspec{FreeSerif}}
\newcommand{\fm}{\fontspec{FreeMono}}
\newcommand{\cmt}{\fontspec{CM Typewriter CE}}
\newcommand{\tnr}{\fontspec{Times New Roman}}
\def\lengthtf{\setlength{\parindent}{25pt}}
\def\lengthzr{\setlength{\parindent}{0pt}}
%\def\UrlFont{\it}
%AR PL KaitiM GB
\date{}
\author{}
\title{\jp なぜ、あなたは Javaオブジェクト指向開発ができないのか}

\makeindex
\begin{document}
\maketitle
\tableofcontents


\chapter*{\jp  はじめに
\newline \cns  前言}

\section*{ \jp なぜ、あなただけJavaで \ruby{オ}{o}\ruby{ブ}{b}\ruby{ジェ}{je}\ruby{ク}{c}\ruby{ト}{t}\ruby{指}{し}\ruby{向}{こう}\ruby{開}{かい}\ruby{発}{はつ}ができないのか
\newline \cns  为什么,只有你做不到Java面向对象开发?}

Javaという\ruby{プ}{p}\ruby{ロ}{ro}\ruby{グ}{g}\ruby{ラ}{ra}\ruby{ミン}{min}\ruby{グ}{g}\ruby{言}{げん}\ruby{語}{ご}\uuline{とともに}「\ruby{オ}{o}\ruby{ブ}{b}\ruby{ジェ}{je}\ruby{ク}{c}\ruby{ト}{t}\ruby{指}{し}\ruby{向}{こう}」という\ruby{考}{かん}え \ruby{方}{かた}が\ruby{一}{いっ}\ruby{般}{ぱん}\ruby{的}{てき}に\ruby{広}{ひろ}まる\uuline{ように}なって、
\ruby{10}{じゅう}\ruby{年}{ねん}\ruby{以}{い}\ruby{上}{じょう}\ruby{立}{た}ちました。
\ruby{時}{とき}を\ruby{同}{おな}じくして\ruby{イン}{in}\ruby{ター}{ter}\ruby{ネッ}{ne}\ruby{ト}{t}が\ruby{普}{ふ}\ruby{及}{きゅう}し、\ruby{サー}{ser}\ruby{バー}{ver}\ruby{サイ}{si}\newline 
\ruby{ド}{de}\ruby{ア}{a}\ruby{プ}{pp}\ruby{リ}{li}\ruby{ケー}{ca}\ruby{ション}{tion}の\ruby{需}{じゅ}\ruby{要}{よう}
が\ruby{増}{ま}したことも\ruby{追}{お}い\ruby{風}{かぜ}となり、\ruby{今}{こん}\ruby{日}{にち}では「オブジェクト指向\uuline{といえば}Java」\uuline{と}\ruby{言}{い}われる\uuline{ほど}、Javaはその\ruby{地}{ち}\ruby{位}{い}を\ruby{確}{かく}\ruby{立}{りつ}しています。\newline
\cns  伴随着Java这种编程语言的普及,“面向对象”这种思考方法也得到广泛应用,已经过了10年多了。同时,因特网也得到普及,Serverside Application的需求也跟风增加了。今天,Java确立了自己的地位,已经达到了只要“说到面向对象就是Java”的程度。
\chapter{\jp  オブジェクト指向をなぜ難しいと感じるのか}

\printindex
\end{document}